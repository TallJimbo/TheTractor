% This file is part of the Tractor project.
% Copyright 2012 David W. Hogg (NYU) and Dustin Lang (Princeton).
% All rights reserved.

% to-do
% -----
% - outline
% - make figures
% - write
% - get comments
% - submit

\documentclass[12pt,pdftex,preprint]{aastex}
\usepackage{amssymb,amsmath,mathrsfs}

\newcommand{\foreign}[1]{\textit{#1}}
\newcommand{\etal}{\foreign{et\,al.}}
\newcommand{\documentname}{\textsl{Note}}
\newcommand{\project}[1]{\textsl{#1}}
\newcommand{\TheTractor}{\project{The~Tractor}}
\newcommand{\sdss}{\project{SDSS}}
\newcommand{\Herschel}{\project{Herschel}}

\newcommand{\tmatrix}[1]{\boldsymbol{#1}}
\newcommand{\inverse}[1]{{#1}^{-1}}
\newcommand{\transpose}[1]{{#1}^{\mathsf T}}
\newcommand{\tvector}[1]{\boldsymbol{#1}}
\newcommand{\pos}{\tvector{x}}
\newcommand{\spos}{\tvector{\xi}}
\newcommand{\mean}{\tvector{m}}
\newcommand{\var}{\tmatrix{V}\!}
\newcommand{\Gm}{\tmatrix{G}}
\newcommand{\Hm}{\tmatrix{H}}
\newcommand{\affine}{\tmatrix{R}}
\newcommand{\uv}{\tvector{u}}
\newcommand{\zero}{\tmatrix{0}}
\newcommand{\identity}{\tmatrix{I}}
\newcommand{\normal}{N}
\newcommand{\given}{\,|\,}
\renewcommand{\star}{\mathrm{star}}
\newcommand{\dev}{\mathrm{dev}}

\newlength{\figwidth}
\setlength{\figwidth}{0.49\textwidth}

\begin{document}

\title{High angular resolution dust density and temperature maps from \Herschel\ imaging data}
\author{Lang, Hogg, etc}
\altaffiltext{1}{To whom correspondence should be addressed; \texttt{foo@bar.edu}}
\altaffiltext{2}{New York University}
\altaffiltext{3}{Max-Planck-Institut f\"ur Astronomie}
\altaffiltext{4}{Princeton University Observatory}

\begin{abstract}
The \Herschel\ Observatory takes images at wavelengths
$100<\lambda<500\,\micron$ and thereby is excellent for measuring the
density and temperature of interstellar dust.  Being
diffraction-limited, however, the images have angular resolution (or
beam size) that is a strong function of wavelength.  In this
\documentname\ we show that it is possible, with very weak priors or
regularization, to infer the properties of the dust---column,
temperature, and emissivity parameter---at the angular resolution of
the highest resolution images.  This is true even though much of the
crucial information comes from the lowest-resolution images.  The
method works by optimizing the posterior probability of a generative
model for the data; the model involves convolution with the
pixel-convolved beam for each image.  The method is demonstrated on a
small patch of \Herschel\ imaging of the dust disk of M31.  The method
can be used in similar circumstances for other \Herschel\ data or on
data from other observatories.  All the code is available under an
open-source license as part of \TheTractor\ image-modeling project.
\end{abstract}

This paper is a stub.  You can edit it and help \TheTractor, so long
as you are either Hogg or Lang!

\acknowledgements
It is a pleasure to thank...

\begin{thebibliography}{70}
\bibitem[Bendinelli(1991)]{bendinelli}
Bendinelli,~O., 1991, \apj, 366, 599
\bibitem[Bendinelli \& Parmeggiani(1995)]{bendinelli2}
Bendinelli,~O. \& Parmeggiani,~G., 1995, \aj, 109, 572
\bibitem[Bovy \etal(2011a)]{xd}
Bovy,~J., Hogg,~D.~W., \& Roweis,~S., 2011, Ann. Appl. Stat., 5, 1657
\bibitem[Bovy \etal(2011b)]{xdqso}
Bovy,~J., \etal, 2011, \apj, 729, 141
\bibitem[Bovy \etal(2012)]{xdqsoz}
Bovy,~J. \etal, 2012, \aj, 749, 41
\bibitem[Cappellari(2002)]{cappellari}
Cappellari,~M., 2002, \mnras, 333, 400
\bibitem[Ciotti \& Bertin(1999)]{ciotti}
Ciotti,~L. \& Bertin,~G., 1999, \aa, 352, 447
\bibitem[Emsellem \etal(1994)]{emsellem}
Emsellem,~E., Monnet,~G., Bacon,~R., \& Nieto,~J.-L., 1994, \aap, 285, 739
\bibitem[Peng \etal(2002)]{galfit}
Peng,~C.~Y., Ho, L.~C., Impey,~C.~D., \& Rix,~H.-W., 2002, \aj, 124, 266
\end{thebibliography}

\end{document}
