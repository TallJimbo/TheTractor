% Copyright 2012 David W. Hogg (NYU).
% All rights reserved.

\documentclass[12pt]{article}

\begin{document}

\section*{Replacing standard galaxy profiles with \\ mixtures of Gaussians}

\noindent
David W. Hogg \\
\textsl{New York University} \\
\textsl{Max-Planck-Institut f\"ur Astronomie} \\[1ex]
Dustin Lang \\
\textsl{Princeton University Observatory}

\begin{abstract}
Exponential, de~Vaucouleurs, and Sersic profiles are simple and
successful models for fitting two-dimensional images of galaxies.  One
numerical issue encountered in this kind of fitting is the pixel
rendering and convolution (or correlation) of the models with the
telescope point-spread function (PSF); these operations are slow, and
easy to get slightly wrong at small radii.  Here we exploit the
realization that these models can be approximated to arbitrary
accuracy with a mixture (linear sum) of two-dimensional Gaussians.
Mixtures of Gaussians are fast to render, fast to affine-transform,
and fast to convolve with mixture-of-Gaussian PSF models, all at
machine precision.  We present worked examples that can be directly
used in image fitting; we are using them ourselves.  We also advocate
modeling PSFs also as arbitrary mixtures of Gaussians.  Amusingly, in
the optically thin limit, a mixture-of-Gaussian two-dimensional model
directly implies its own three-dimensional de-projection.
\end{abstract}

...Some other authors have worked with mixtures of Gaussians...Very
related work...thinking more generally...Here we want to solve a very
specific set of numerical problems...

...Mixtures of Gaussians sometimes used for PSF fitting...Used in
XD...Other places?...

...Rendering of Sersic profiles (of which exp and dev are types) is
non-trivial at the center...for the same reason convolution is a
bitch...

...If we wanted to approximate these models, what objective function
would we use?  What tolerances are okay?...

...What do we get by running the relevant approximations for exp and
dev?...Can we make a continuous approximation for the Sersic
profiles?...

...How to use these in a real setting...An example from the Tractor...

\end{document}
